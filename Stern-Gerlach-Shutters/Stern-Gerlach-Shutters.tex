% -*- mode: ConTeXt; encoding: utf-8; -*-
%
% This is free and unencumbered software released into the public domain.
%
% Anyone is free to copy, modify, publish, use, compile, sell, or
% distribute this software, either in source code form or as a compiled
% binary, for any purpose, commercial or non-commercial, and by any
% means.
%
% In jurisdictions that recognize copyright laws, the author or authors
% of this software dedicate any and all copyright interest in the
% software to the public domain. We make this dedication for the benefit
% of the public at large and to the detriment of our heirs and
% successors. We intend this dedication to be an overt act of
% relinquishment in perpetuity of all present and future rights to this
% software under copyright law.
%
% THE SOFTWARE IS PROVIDED "AS IS", WITHOUT WARRANTY OF ANY KIND,
% EXPRESS OR IMPLIED, INCLUDING BUT NOT LIMITED TO THE WARRANTIES OF
% MERCHANTABILITY, FITNESS FOR A PARTICULAR PURPOSE AND NONINFRINGEMENT.
% IN NO EVENT SHALL THE AUTHORS BE LIABLE FOR ANY CLAIM, DAMAGES OR
% OTHER LIABILITY, WHETHER IN AN ACTION OF CONTRACT, TORT OR OTHERWISE,
% ARISING FROM, OUT OF OR IN CONNECTION WITH THE SOFTWARE OR THE USE OR
% OTHER DEALINGS IN THE SOFTWARE.

\define\eblack{{{\rm e}^{\rm b}}}
\define\ewhite{{{\rm e}^{\rm w}}}
\define\ebew{{(\eblack,\ewhite)}}
\define\eweb{{(\ewhite,\eblack)}}

\enableregime[utf]
\frenchspacing
\mainlanguage[en]

\setupbodyfont[schoolbook, 11pt]
\setupbodyfontenvironment[default][em=italic]

\setuppapersize[letter][letter]

\setuplayout
  [location=singlesided,
    width=fit,
    backspace=1.25in]

\setuppagenumbering
  [alternative=singlesided,
    location={footer, middle},
    way=bytext]

\setuphead
  [title]
  [align=middle]
\setuphead
  [subject]
  [align=middle,
    commandbefore={---\hskip0.6em},
    commandafter={\hskip0.5em---}]

\starttext

\title{Stern-Gerlach Shutters}

\subject{The Problem}

\startitemize[n]

\item The Stern-Gerlach Scientific Contraption Company makes a device
  called a Stern-Gerlach Shutter. One of these is depicted
  in~\in{Figure}[fig:shutter].

  \placefigure[][fig:shutter]{Stern-Gerlach Shutter}{\externalfigure[Stern-Gerlach-Shutter-2.png][width=2.45in]}

\item The settings gauge at the bottom of the device is set at a
  desired angle between zero and~$\pi$. The shutter sheet slides
  horizontally, guided by a long bar that is connected to the rotating
  gauge, and which can slide vertically through a tube at the left
  edge of the shutter sheet.

\item The shutter sheet is used to cover a rectangular opening that is
  centered over the settings gauge’s pivot. The width of the opening
  is equal to the diameter of the semicircle swept out by the settings
  gauge.

\item We have two of these Stern-Gerlach Shutters, and for each one a
  corresponding Stern-Gerlach Shutter Shooter. A Stern-Gerlach Shutter
  Shooter fires a pellet at a random point inside the boundaries of
  the opening of a Stern-Gerlach Shutter. The pellet may either pass
  through the opening or bounce off the shutter sheet. The Shutter
  Shooter may fire either a black pellet or a white pellet, at the
  discretion of the experimenter.

\item Shutter~1 is set to~$\phi_1$ and~Shutter~2 is set to~$\phi_2$.

\item According to the flip of a fair coin, we shoot either a black
  pellet at~Shutter~1 and a white pellet at~Shutter~2, or the other
  way around.

\item Let~$+1$ represent either a black pellet passing through an
  opening or a white pellet bouncing off a shutter sheet. Let~$-1$
  represent pellets doing it the other way around.

\item Compute, for the two Stern-Gerlach shutters, the correlation
  coefficient~$\rho$ as a function of~$\phi_1-\phi_2$.

\stopitemize

\startalignment[middle]
  *~*~*
\stopalignment

\page

\subject{The Solution}

\startitemize[n]

\item \in{Figure}[fig:shutterannot] depicts important dimensions of a
  Stern-Gerlach Shutter.

  \placefigure[][fig:shutterannot]{Dimensions of a Stern-Gerlach
    Shutter}{\externalfigure[Stern-Gerlach-Shutter-annotated-2.png][width=2.55in]}

\item For convenience, let the radius of the settings gauge equal
  one. Therefore the diameter equals two and the width of the opening
  is also two.

\item Refer to the settings gauge’s setting as~$\phi$. The projection
  of the gauge onto horizontal is~$\cos\phi$.

\item By double-angle identities in the {\it CRC Handbook of
  Mathematical Sciences},
  \startformula
    \cos\phi=\cos^2(\phi/2)-\sin^2(\phi/2)=2\cos^2(\phi/2)-1=1-2\sin^2(\phi/2)
  \stopformula

\item As shown in~\in{Figure}[fig:shutterannot], these identities
  imply the shutter sheet divides the width of the opening
  into~$2\cos^2(\phi/2)$ and~$2\sin^2(\phi/2)$.

\item Thus the shutter sheet divides the area of the opening into
  proportions~$\cos^2(\phi/2)$ and~$\sin^2(\phi/2)$.

\item The probability of a pellet passing through the opening is
  therefore~$\cos^2(\phi/2)$, and the probability of it bouncing off
  the shutter sheet is~$\sin^2(\phi/2)$.

\item Let us solve for the case~$\phi_1=\phi_{01}\in[0,\pi]$,
  $\phi_2=\phi_{02}=0$. Shutter~2 is fully open, so one merely counts
  black pellets as~$+1$ and white pellets as~$-1$. It is as if there
  were no shutter sheet at all, and in fact one need not bother
  shooting at~Shutter~2.

\item For this special case,
  \startformula
    \rho=\rho^{++}+\rho^{+-}+\rho^{-+}+\rho^{--}
  \stopformula
  where
  \startformula
    \startmathalignment [n=2, align={right,left}]
      \NC\rho^{++}\NC=(+1)(+1)\sin^2(\phi_{01}/2)/2\NR
      \NC\rho^{+-}\NC=(+1)(-1)\cos^2(\phi_{01}/2)/2\NR
      \NC\rho^{-+}\NC=(-1)(+1)\cos^2(\phi_{01}/2)/2\NR
      \NC\rho^{--}\NC=(-1)(-1)\sin^2(\phi_{01}/2)/2\NR
    \stopmathalignment
  \stopformula
  The calculation above is for the covariance. The correlation
  coefficient is covariance with its range normalized to~$[-1,+1]$,
  but this covariance already has that.

\item Using the double-angle identities again, $\rho$~simplifies to
  \startformula
    \rho = -(\cos^2\frac{\phi_{01}}{2}-\sin^2\frac{\phi_{01}}{2})
    =-\cos\phi_{01}=-\cos(\phi_{01}-\phi_{02})
  \stopformula

\item Suppose one is given~$\phi_1$ and~$\phi_2$ as the respective
  settings. Assume~$\phi_1\ge\phi_2$. (Otherwise you can simply
  reverse their roles, because cosine is symmetric around the origin.)
  Then let~$\phi_{01}=\phi_1-\phi_2$ and~$\phi_{02}=0=\phi_2-\phi_2$,
  and add~$\phi_2$ to both sides of the minus sign
  in~$\phi_{01}-\phi_{02}$.
  \startformula
    \startmathalignment [n=2, align={right,left}]
      \NC\rho\NC=-\cos(\phi_{01}-\phi_{02}) \NR
      \NC    \NC=-\cos((\phi_1-\phi_2)-(\phi_2-\phi_2)) \NR
      \NC    \NC=-\cos(\phi_1-\phi_2) \NR
    \stopmathalignment
  \stopformula

\item And there we have it, the general solution:
  $\rho = -\cos(\phi_1-\phi_2)$. It is invariant under in-unison
  rotations of the settings.

\stopitemize


\page

\subject{Alternative Solution by Quantum Mechanics}

\startitemize[n]

\item A solution by quantum mechanics is much more complicated, so no
  wonder quantum mechanics is considered arcane and difficult. Because
  all mathematical methods\footnote{{\it When employed correctly\/}!}
  must reach the same result, quantum mechanics is never necessary to
  solve a problem. Thus one might consider not using it.

\item Nevertheless, let us use it here. I have not had coursework in
  quantum mechanics and so will try to figure this out for
  myself. Otherwise, one may look up an analysis of the EPR-Bohm
  problem for Stern-Gerlach magnets. A similar analysis ought to apply
  to Stern-Gerlach Shutters.

\item Possible states of our calculation will form a linear space of
  ordered pairs of vectors. We attribute no physical meaning to this
  space. The ordered pairs in the space represent states in a symbolic
  calculation of the correlation coefficient.

\item Either of the two pellets has its own linear space that is more
  primitive than the ‘full’ linear space. Let~$\eblack$ represent the
  basis vector for a black pellet and~$\ewhite$ the basis vector for a
  white pellet. These vectors induce a linear space. So-called
  ‘superpositions’ can be formed from linear combinations of the two
  basis vectors.

\item However, for a linear combination to be a valid ‘superposition’,
  there is a constraint on it.  The scalar multipliers of the
  components may be real or complex numbers, but the sums of the
  squares of the absolute values of those multipliers must equal one.

\item Now, let us clarify that the state space represents {\it the
  state of a calculation}, and {\it not\/} the physical state of the
  experiment. There is no constraint on the shooting of~Pellet~1
  and~Pellet~2 being done simultaneously. If a settings gauge be set
  at zero, no shooting is necessary at all.\footnote{Similar
    observations are true of experiments with photons, electrons,
    etc., although physicists have not realized this. Conservation
    laws, not ‘superposition’, account for particles occurring in
    pairs.} Nevertheless, the result of the flip of the fair coin will
  be the superposition~$e^{i\pi/4}\ebew+e^{i\pi/4}\eweb$. We

% \item Actually, hardly anyone one knows what a Stern-Gerlach Shutter
%   does ‘quantum mechanically’, except the wizards at the Stern-Gerlach
%   company. They inform us that a Stern-Gerlach Shutter set to~$\phi$
%   multiplies~$\eblack$ by~$e^{+i\phi/2}$ and~$\ewhite$
%   by~$e^{-i\phi/2}$. Call this operation the ‘phase shift’. Then the
%   pellet passes through the opening with probability equal to the
%   square of the projection of the new vector onto the basis
%   vector.\footnote{One can deduce these operations from intermediate
%     results of the earlier solution. Now deduce the other way: a
%     Stern-Gerlach magnet is a kind of shutter or valve, whose
%     effective area is the square of a cosine or sine. Therefore
%     problems involving Stern-Gerlach magnets and ‘spin’ can be solved
%     without quantum mechanics---that is, without
%     ‘superpositions’. There is no ‘entanglement’. There is no ‘action
%     at a distance’. Superpositions are not a property of the problem,
%     but of the method of solution---and it is impossible for a method
%     of solution to be mandatory except by authoritarian fiat.}

\item NOPE. I DON’T KNOW HOW TO DO THIS, AND IT IS IMMENSELY
  COMPLICATED. I GIVE UP.~$\square$

\stopitemize

\stoptext
