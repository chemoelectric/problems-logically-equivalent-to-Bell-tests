% -*- mode: ConTeXt; encoding: utf-8; -*-
%
% This is free and unencumbered software released into the public domain.
%
% Anyone is free to copy, modify, publish, use, compile, sell, or
% distribute this software, either in source code form or as a compiled
% binary, for any purpose, commercial or non-commercial, and by any
% means.
%
% In jurisdictions that recognize copyright laws, the author or authors
% of this software dedicate any and all copyright interest in the
% software to the public domain. We make this dedication for the benefit
% of the public at large and to the detriment of our heirs and
% successors. We intend this dedication to be an overt act of
% relinquishment in perpetuity of all present and future rights to this
% software under copyright law.
%
% THE SOFTWARE IS PROVIDED "AS IS", WITHOUT WARRANTY OF ANY KIND,
% EXPRESS OR IMPLIED, INCLUDING BUT NOT LIMITED TO THE WARRANTIES OF
% MERCHANTABILITY, FITNESS FOR A PARTICULAR PURPOSE AND NONINFRINGEMENT.
% IN NO EVENT SHALL THE AUTHORS BE LIABLE FOR ANY CLAIM, DAMAGES OR
% OTHER LIABILITY, WHETHER IN AN ACTION OF CONTRACT, TORT OR OTHERWISE,
% ARISING FROM, OUT OF OR IN CONNECTION WITH THE SOFTWARE OR THE USE OR
% OTHER DEALINGS IN THE SOFTWARE.

\enableregime[utf]
\frenchspacing
\mainlanguage[en]

\setupbodyfont[schoolbook, 11pt]
\setupbodyfontenvironment[default][em=italic]

\setuppapersize[letter][letter]

\setuplayout
  [location=singlesided,
    width=fit,
    backspace=1.25in]

\setuppagenumbering
  [alternative=singlesided,
    location={footer, middle},
    way=bytext]

\setuphead
  [title]
  [align=middle]
\setuphead
  [subject]
  [align=middle,
    commandbefore={---\hskip0.6em},
    commandafter={\hskip0.5em---}]

\starttext

\title{Two Oscillators}

\subject{The Problem}

\startitemize[n]
\item You have two oscillators,~$a(t)=\sin(\omega t + \phi_a)$
  and~$b(t)=\sin(\omega t+\phi_b)$, where
  \startformula
    (\phi_a,\phi_b) =
    \startcases[align={left,left}, distance=1em]
      \NC(0, \frac{\pi}{2}) \NC{\rm with\ probability\ }\frac{1}{2} \NR
      \NC(\frac{\pi}{2}, 0) \NC{\rm with\ probability\ }\frac{1}{2} \NR
    \stopcases
  \stopformula
\item The oscillator outputs are subjected to phase shifts,
  giving~$a'(t)=\sin(\omega t + \phi_a + \theta_a)$
  and~$b'(t)=\sin(\omega t + \phi_b + \theta_b)$.
\item Define the function
  \startformula
    g(x)=
    \startcases[align={left,left}, distance=1em]
      \NC +1 \NC{\rm with\ probability\ } x^2 \NR
      \NC -1 \NC{\rm with\ probability\ } 1 - x^2 \NR
    \stopcases
  \stopformula
\item Calculate the correlation coefficient of~$g(a'(t))$
  and~$g(b'(t))$ as a function of\kern1pt~$\theta_a-\theta_b$.
\stopitemize

\subject{The Solution}

\startitemize[n]
\item By change of the time variable and
  letting~$\theta_{ab}=\theta_a-\theta_b$, one can rewrite~$a'(t)$
  and~$b'(t)$ as
  \startformula
      \NC(\alpha(\tau),\beta(\tau))=
    \startcases
      \NC(\sin(\omega\tau + \theta_{ab}), \cos\,\omega\tau)
      \NC{\rm with\ probability\ }\frac{1}{2} \NR
      \NC(\cos(\omega\tau + \theta_{ab}), \sin\,\omega\tau)
      \NC{\rm with\ probability\ }\frac{1}{2} \NR
    \stopcases
  \stopformula

\item At $\tau=0$ this gives
  \startformula
      \NC(\alpha(0),\beta(0))=
    \startcases
      \NC(\sin\theta_{ab}, 1)
      \NC{\rm with\ probability\ }\frac{1}{2} \NR
      \NC(\cos\theta_{ab}, 0)
      \NC{\rm with\ probability\ }\frac{1}{2} \NR
    \stopcases
  \stopformula

\item The correlation coefficient for~$\tau=0$, in terms of the
  difference of angles~$\theta_{ab}$ and~zero,
  $\theta_{ab}-0=\theta_{ab}$, is
  \startformula
    \rho=\rho^{++}+\rho^{+-}+\rho^{-+}+\rho^{--}
  \stopformula
  where
\startformula
  \startmathalignment [n=2, align={right,left}]
    \NC\rho^{++}\NC=\frac{1}{2}(+1)(+1)\sin^2\theta_{ab}\NR
    \NC\rho^{+-}\NC=\frac{1}{2}(+1)(-1)\cos^2\theta_{ab}\NR
    \NC\rho^{-+}\NC=\frac{1}{2}(-1)(+1)\cos^2\theta_{ab}\NR
    \NC\rho^{--}\NC=\frac{1}{2}(-1)(-1)\sin^2\theta_{ab}\NR
  \stopmathalignment
\stopformula

\item Using a double-angle identity, this simplifies to
  \startformula
    \rho=-(\cos^2\theta_{ab}-\sin^2\theta_{ab})=-\cos\,2\theta_{ab}
  \stopformula
  or, in other words,
  \startformula
    \rho=-\cos\,2(\theta_a-\theta_b)
  \stopformula
  which would seem to be the answer to the problem. However, we
  derived it under the assumption that~$\tau=0$.

\item Note, though, that the solution is invariant under in-unison
  rotations of\kern1pt~$\theta_a$ and~$\theta_b$. Any change of the
  time variable is absorbed by such a rotation. Therefore~$\rho$ is
  indeed the general solution to the problem.\footnote{It is also, as
    it must be, the correlation coefficient calculated by quantum
    mechanics for a two-channel optical Bell test experiment. The two
    problems are equivalent, and all valid methods of solution must
    come to the same result. Be forewarned, though: ‘doctors of
    philosophy’ might suggest the problems {\it are not\/} equivalent,
    for reasons beyond the realm of mathematics. For instance, the
    oscillators were not created by a calcium radiative cascade and
    therefore were not ‘quantum’. If they {\it had been\/} created by
    cascade, {\it then\/} only quantum mechanics (they might suggest)
    could have solved the problem. This would be circular argument,
    however, because the Bell test is what supposedly ‘proves’ that
    the cascade is ‘quantum’! It is an argument enforceable ({\it and
      in the actual world enforced\/}) by authority, not logic. There
    is actually no special quality associated with ‘quantum’
    physics. ‘Quantum mechanics’ refers to a patchwork of obfuscated
    methods for solving physics problems. Here we solved the same
    problem by other, in this case better means.}~$\square$

\stopitemize

\stoptext
