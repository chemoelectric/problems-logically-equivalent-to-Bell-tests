% -*- mode: ConTeXt; encoding: utf-8; -*-
%
% This is free and unencumbered software released into the public domain.
%
% Anyone is free to copy, modify, publish, use, compile, sell, or
% distribute this software, either in source code form or as a compiled
% binary, for any purpose, commercial or non-commercial, and by any
% means.
%
% In jurisdictions that recognize copyright laws, the author or authors
% of this software dedicate any and all copyright interest in the
% software to the public domain. We make this dedication for the benefit
% of the public at large and to the detriment of our heirs and
% successors. We intend this dedication to be an overt act of
% relinquishment in perpetuity of all present and future rights to this
% software under copyright law.
%
% THE SOFTWARE IS PROVIDED "AS IS", WITHOUT WARRANTY OF ANY KIND,
% EXPRESS OR IMPLIED, INCLUDING BUT NOT LIMITED TO THE WARRANTIES OF
% MERCHANTABILITY, FITNESS FOR A PARTICULAR PURPOSE AND NONINFRINGEMENT.
% IN NO EVENT SHALL THE AUTHORS BE LIABLE FOR ANY CLAIM, DAMAGES OR
% OTHER LIABILITY, WHETHER IN AN ACTION OF CONTRACT, TORT OR OTHERWISE,
% ARISING FROM, OUT OF OR IN CONNECTION WITH THE SOFTWARE OR THE USE OR
% OTHER DEALINGS IN THE SOFTWARE.

\enableregime[utf]
\frenchspacing
\mainlanguage[en]

\setupbodyfont[schoolbook, 11pt]
\setupbodyfontenvironment[default][em=italic]

\setuppapersize[letter][letter]

\setuplayout
  [location=singlesided,
    width=fit,
    backspace=1.25in]

\setuppagenumbering
  [alternative=singlesided,
    location={footer, middle},
    way=bytext]

\setuphead
  [title]
  [align=middle]
\setuphead
  [subject]
  [align=middle,
    commandbefore={---\hskip0.6em},
    commandafter={\hskip0.5em---}]

\starttext

\title{Two Oscillators}

\subject{The Problem}

\startitemize[n]
\item You have two oscillators,~$a(t)=\sin(\omega t + \phi_a)$
  and~$b(t)=\sin(\omega t+\phi_b)$, where
\startformula
  (\phi_a,\phi_b) =
  \startcases[align={left,left}, distance=1em]
    \NC(0, \frac{\pi}{2}) \NC{\rm with\ probability\ }\frac{1}{2} \NR
    \NC(\frac{\pi}{2}, 0) \NC{\rm with\ probability\ }\frac{1}{2} \NR
  \stopcases
\stopformula
\item The oscillator outputs are subjected to phase shifts,
  giving~$a'(t)=\sin(\omega t + \phi_a + \theta_a)$
  and~$b'(t)=\sin(\omega t + \phi_b + \theta_b)$.
\item Define the function
  \startformula
    g(x)=
    \startcases[align={left,left}, distance=1em]
      \NC +1 \NC \NC{\rm with\ probability\ } x^2 \NR
      \NC -1 \NC \NC{\rm with\ probability\ } 1 - x^2 \NR
    \stopcases
  \stopformula
\item Calculate the correlation coefficient of~$g(a'(t))$
  and~$g(b'(t))$ as a function of\kern1pt~$\theta_a-\theta_b$.

% \item You have one box of Chocolate Cookies and one box of Vanilla
%   Cookies.
% \item You will deliver one of the boxes to Alice and one of the boxes
%   to Bob. Neither has a preference of flavor.
% \item You flip a coin to decide who gets which box.\footnote{This step
%   is equivalent to a light source generating a pair of perpendicularly
%   polarized photons.}
% \item Alice has a favorite angle,~$\alpha$. Bob also has a favorite
%   angle,~$\beta$. You write down the numbers $\cos^2\alpha$,
%   $\sin^2\alpha$, $\cos^2\beta$, and~$\sin^2\beta$. Also, you have a
%   pair of ten-sided dice. You can roll these dice to get a number
%   between~$0.00$ and~$0.99$, inclusive.
% \item Suppose Alice is getting the box of Chocolate Cookies. You roll
%   the dice. If the number you roll is less than
%   $\cos^2\alpha$, then you write~$+1$ on the box. Otherwise
%   you write~$-1$.\footnote{This is equivalent to a photon being
%     detected in one of the channels of a polarizing beam splitter.}
% \item Suppose, on the other hand, that Alice is getting the box of
%   Vanilla Cookies. You roll the dice.  If the number you roll is less
%   than $\sin^2\alpha$, then you write~$+1$ on the
%   box. Otherwise you write~$-1$.
% \item Suppose Bob is getting the box of Chocolate Cookies. You roll
%   the dice. If the number you roll is less than
%   $\cos^2\beta$, then you write~$+1$ on the box. Otherwise
%   you write~$-1$.
% \item Suppose, on the other hand, that Bob is getting the box of
%   Vanilla Cookies. You roll the dice.  If the number you roll is less
%   than $\sin^2\beta$, then you write~$+1$ on the
%   box. Otherwise you write~$-1$.
% \item Calculate the correlation coefficient of the numbers written on
%   the cookie boxes, as a function of~$\alpha-\beta$.\footnote{The
%     answer from quantum mechanics is~$-\!\cos\,2(\alpha-\beta)$, and
%     physicists assure us only quantum mechanical systems can achieve
%     this solution. According to physicists, even logically equivalent
%     systems cannot do so, because they have no entangled superposition
%     states nor action at a distance!}
\stopitemize

\subject{The Solution}

% First let us solve the problem for $\alpha_0=\alpha-\beta$
% and~$\beta_0=0$. Then
% \startformula
%   \startmathalignment [n=2, align={right,left}]
%     \NC\cos^2\beta_0\NC= 1 \NR
%     \NC\sin^2\beta_0\NC= 0 \NR
%   \stopmathalignment
% \stopformula
% which simplifies things. The choice of markings as $\{+1,-1\}$ makes
% the correlation coefficient even simpler to compute. It is
% \startformula
% \rho_0=\rho_0^{++}+\rho_0^{+-}+\rho_0^{-+}+\rho_0^{--}
% \stopformula
% where
% \startformula
%   \startmathalignment [n=2, align={right,left}]
%     % The roll for Bob always succeeds for Chocolate, always fails for
%     % Vanilla.
%     \NC\rho_0^{++}\NC=\frac{1}{2}(+1)(+1)\sin^2\alpha_0\NR % Vanilla/Choc
%     \NC\rho_0^{+-}\NC=\frac{1}{2}(+1)(-1)\cos^2\alpha_0\NR % Choc/Vanilla
%     \NC\rho_0^{-+}\NC=\frac{1}{2}(-1)(+1)\cos^2\alpha_0\NR % Vanilla/Choc
%     \NC\rho_0^{--}\NC=\frac{1}{2}(-1)(-1)\sin^2\alpha_0\NR % Choc/Vanilla
%   \stopmathalignment
% \stopformula
% Thus (by a double-angle identity found in reference books)
% \startformula
%   \rho_0=-(\cos^2\alpha_0-\sin^2\alpha_0)=-\cos\,2\alpha_0
% \stopformula
% But~$\alpha_0=\alpha-\beta$, so~$\rho_0$ is the desired
% solution.\footnote{This solution is the same as the one from quantum
%   mechanics. But be forewarned: there are many ‘doctors of philosophy’
%   who would object to my solution yet are permitted to teach
%   university classes.}  Rename it~$\rho$ without the subscript. The
% solution is
% \startformula
%   \rho=-\cos\,2(\alpha-\beta)
% \stopformula
% Interestingly, although the particular ‘scattering’ of~$+1$ and~$-1$
% marks (were a person to do the operation many times) depends on the
% specific values of~$\alpha$ and~$\beta$, the correlation coefficient
% depends only on the difference~$\alpha-\beta$. That is, there exists a
% relationship between~$+1$ and~$-1$ marks that is invariant under
% in-unison rotation of the angles.\footnote{Of course, the same rotational
%   invariance was always true of Bell test experiments, but little
%   remarked on.}~$\square$

\stoptext
