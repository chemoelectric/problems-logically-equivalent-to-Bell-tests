% -*- mode: ConTeXt; encoding: utf-8; -*-

\enableregime[utf]
\frenchspacing
\mainlanguage[en]

\setupbodyfont[schoolbook, 11pt]
\setupbodyfontenvironment[default][em=italic]

\definepapersize[Cookies][letter]
\setuppapersize[Cookies][letter]
\setuplayout[
  location=singlesided,
  backspace=1.25in,
  width=fit,
  grid=yes
]

\setuppagenumbering[
  alternative=singlesided,
  location={footer, middle},
  way=bytext
]

\starttext

\title{Cookies for Alice and Bob}

\subject{The Problem}

\startitemize[n]
\item You have one box of Chocolate Cookies and one box of Vanilla
  Cookies.
\item You will deliver one of the boxes to Alice and one of the boxes
  to Bob. Neither has a preference of flavor.
\item You flip a coin to decide who gets which box.\footnote{This step
  is equivalent to a light source generating a pair of perpendicularly
  polarized photons.}
\item Alice has a favorite angle,~$\alpha$. Bob also has a favorite
  angle,~$\beta$. You write down the numbers $\cos^2\alpha$,
  $\sin^2\alpha$, $\cos^2\beta$, and~$\sin^2\beta$. Also, you have a
  pair of ten-sided dice. You can roll these dice to get a number
  between~$0.00$ and~$0.99$, inclusive.
\item Suppose Alice is getting the box of Chocolate Cookies. You roll
  the dice. If the number you roll is less than
  $\cos^2\alpha$, then you write~$+1$ on the box. Otherwise
  you write~$-1$.\footnote{This is equivalent to a photon being
    detected in one of the channels of a polarizing beam splitter.}
\item Suppose, on the other hand, that Alice is getting the box of
  Vanilla Cookies. You roll the dice.  If the number you roll is less
  than $\sin^2\alpha$, then you write~$+1$ on the
  box. Otherwise you write~$-1$.
\item Suppose Bob is getting the box of Chocolate Cookies. You roll
  the dice. If the number you roll is less than
  $\cos^2\beta$, then you write~$+1$ on the box. Otherwise
  you write~$-1$.
\item Suppose, on the other hand, that Bob is getting the box of
  Vanilla Cookies. You roll the dice.  If the number you roll is less
  than $\sin^2\beta$, then you write~$+1$ on the
  box. Otherwise you write~$-1$.
\item Calculate the correlation coefficient of the numbers written on
  the cookie boxes, as a function of~$\alpha-\beta$.\footnote{The
    answer from quantum mechanics is~$-\cos\{2(\alpha-\beta)\}$, and
    physicists assure us only quantum mechanical systems can achieve
    this correlation. According to physicists, not even logically
    equivalent systems can do so!}
\stopitemize

\subject{The Solution}

First let us solve the problem for $\alpha=\alpha_0$
and~$\beta=\beta_0=0$. We will use~$\beta_0$ as an {\em angular
  origin}. Then
\startformula
  \startmathalignment [n=2, align={right,left}]
    \NC\cos^2\alpha\NC=\cos^2\alpha_0 \NR
    \NC\sin^2\alpha\NC=\sin^2\alpha_0 \NR
    \NC\cos^2\beta\NC= 1 \NR
    \NC\sin^2\beta\NC= 0 \NR
  \stopmathalignment
\stopformula
The choice of markings as $\{+1,-1\}$ makes the correlation
coefficient simple to compute. It is
\startformula
\rho_0=\rho_0^{++}+\rho_0^{+-}+\rho_0^{-+}+\rho_0^{--}
\stopformula
where
\startformula
  \startmathalignment [n=2, align={right,left}]
    % The roll for Bob always succeeds for Chocolate, always fails for
    % Vanilla.
    \NC\rho_0^{++}\NC=\frac{1}{2}(+1)(+1)\sin^2\alpha_0\NR % Vanilla/Choc
    \NC\rho_0^{+-}\NC=\frac{1}{2}(+1)(-1)\cos^2\alpha_0\NR % Choc/Vanilla
    \NC\rho_0^{-+}\NC=\frac{1}{2}(-1)(+1)\cos^2\alpha_0\NR % Vanilla/Choc
    \NC\rho_0^{--}\NC=\frac{1}{2}(-1)(-1)\sin^2\alpha_0\NR % Choc/Vanilla
  \stopmathalignment
\stopformula
Thus (by a double-angle identity found in reference books)
\startformula
  \rho_0=-(\cos^2\alpha_0-\sin^2\alpha_0)=-\cos(2\alpha_0)
\stopformula
Let~$\alpha=\alpha_0+\beta$. Then~$\rho_0$ is the desired solution, so
rename it~$\rho$ without the subscript:
\startformula
  \rho=-\cos\{2(\alpha-\beta)\}
\stopformula
Although the particular arrangement of~$+1$ and~$-1$ marks depends on
the specific values of~$\alpha$ and~$\beta$, the correlation
coefficient depends only on their difference.\footnote{This solution
  is the same as the one from quantum mechanics. Presumably some
  physicists would Gish Gallop a lot of objections to my
  solution. Such persons are allowed to teach classes?} Of course the
corresponding fact was always true of Bell test experiments.

\stoptext
